\documentclass{ctexart}
\usepackage{enumerate}
\usepackage[colorlinks,linkcolor=blue]{hyperref}

\usepackage{ulem} 
\usepackage{caption}
\usepackage{graphicx, subfig}



\title{霓虹歌姬企画}
\author{南方科技大学日语社}
\begin{document}
\maketitle
\tableofcontents

\section{前言}
前几天日语社的同学们一起去会展中心的伊势海老这个店吃日料,本来想着要不然跟服务员光说日语吧,看看会发生什么事。但是到了店里却觉得店员应该还是不可能懂日语的,遂作罢。我觉得如果想要日语炸店的话,还是要去些人均1000以上的店。虽说店员们不会说日语,可是店里依然飘荡着日语歌。几首歌听下来,觉得还挺好听的,就是我对这些歌曲和歌手都没有什么了解。同学们指出这是某某有名的歌手唱的有名的歌,我才猛然发现我虽然平素也是主要听日本的歌曲,但是对于ACG之外的日本歌曲其实并没有太多的了解。之前社里一起看红白歌会,回忆起廖同学讲着讲那讲得头头是道,加上我有点歌荒,不禁觉得是时候扩展一下我的听歌范围了。之前的我连安室奈美惠都不知道,多亏廖同学指点才算是知道了这么个人名。听歌就像看论文一样,看得多了写个综述是很好的。综述这种东西既有利于自己总结,也便于跟别人分享。遂决定组织各位写这么一个小小的综述集子,来看看日本到底有多少值得注意的歌手。每个词条大概有一张图片,再配上编写者的一些说明,最后给新人一个听歌的顺序,使得新人可以循序渐进地了解与习惯这个歌手的风格。歌手按虚拟偶像,声优,ACG歌手和非ACG歌手分类,方便读者检索。\\
“霓虹歌姬企画”这个名字是我拍脑袋定的。之所以叫歌姬是因为觉得这样叫比较萌。总之就是这样一个集子,感谢各位的贡献。——张子健


\section{虚拟偶像}
\subsection{$\mu$'s}
\begin{figure}[h]
\centering
 \includegraphics[width=1.0\textwidth]{lovelive.jpg}
 \caption{LoveLive!}
\end{figure}
$\mu$'s是LoveLive!企划里校园偶像组合的名字。“为了拯救濒临废校的学校,九位少女决定组成校园偶像”。我觉得校园偶像这个设定还是很好玩的。我刚入学的时候曾经想要为了\sout{跟妹子玩耍}宣传南科大,在学校里组织一个校园偶像团体。可是实在是找不到合适的妹子,遂作罢。\\
与同一类型的偶像大师相比,虽说LoveLive!的动画做得显然没有偶像大师765要好,但是歌还是很好听的。我个人认为LoveLive!的音乐更胜一筹。其音乐人藤泽庆昌也是少女歌剧和宝石之国的音乐人。
\subsubsection*{歌曲推荐}
实际上$\mu$'s的曲风还是比较多变的。在这里列出一些笔者喜欢的歌。其中soldier game每次去KTV都想唱...
\begin{enumerate}
\item No brand girls
\item soldier game
\item 春情ロマンティック
\item Music S.T.A.R.T!!
\item 僕らのLIVE 君とのLIFE
\end{enumerate}




\end{document}